\documentclass[11pt]{article}
\usepackage[paperwidth=8.5in, paperheight=11in]{geometry}

\usepackage{../tjimo}
%\usepackage[pdftex]{graphicx}
\usepackage{asymptote}

\begin{comment}
\def\answer{\comment}
\def\solution{\comment}
\def\solutionone{\comment}
\def\solutiontwo{\comment}
\end{comment}

\newcommand{\sevenpoints}{Time limit: 60 minutes.}
\newcommand{\righthead}{\fdbox{Round}{Individual}}

\begin{document}
\setlength{\parindent}{0pt}

\begin{problem}
Compute $2^0 + 1 \cdot 7 + 2(0 \cdot 1 + 8)$.
\end{problem}

\begin{answer}
$\boxed{24}$
\end{answer}

\begin{solution}
We follow the order of operations to obtain
\begin{align*}
2^0 + 1 \cdot 7 + 2(0 \cdot 1 + 7) &= 1 + 7 + 2(0 + 8) \\
&= 8 + 2(7) \\ 
&= 8 + 16 \\
&= \boxed{24}.
\end{align*}
\end{solution}


\begin{problem}
NAME turned $14$ years old in the year 2017. Determine in what year NAME's age will double.
\end{problem}

\begin{answer}
$\boxed{2031}$
\end{answer}

\begin{solution}
NAME's age will double in $14$ years, or in the year $2017 + 14 = \boxed{2031}$.
\end{solution}


\begin{problem}
A fence encloses a square region with area $2025\ \text{ft}^2$. Compute the perimeter of the fence, in ft.
\end{problem}

\begin{answer}
$\boxed{180}$ (ft)
\end{answer}

\begin{solution}
The area of a square with side length $s$ is $s^2$, so the length of one side enclosed by the fence is
$\sqrt{2025\ \text{ft}^2} = 45\ \text{ft}$. There are four sides, so the perimeter is four times the side
length, which is $4 \cdot 45\ \text{ft} = \boxed{180}\ \text{ft}$.
\end{solution}


\begin{problem}
Today is Saturday, October 28, 2017. Determine the day of the week on which Halloween falls \textit{next} year
(October 31, 2018).
\end{problem}

\begin{answer}
$\boxed{\text{Wednesday}}$
\end{answer}

\begin{solution}
Since 2018 is not a leap year, October 28, 2018 is $365$ days after October 28, 2017. As $365 = 7(52)+1$ is
$1$ more than a multiple of $7$, October 28, 2018 will be one day after Saturday, or Sunday.
October 31 is $3$ more days later, which is $\boxed{\text{Wednesday}}$.
\end{solution}


\begin{problem}
Compute the number of positive three-digit integers without leading zeros (so 042 does not count) that
have either three even digits or three odd digits.
\end{problem}

\begin{answer}
$\boxed{225}$
\end{answer}

\begin{solution}%[Kyle]
If the number has three even digits, then the hundreds digit can be either $2, 4, 6$, or $8$, for $4$ choices,
and the tens and ones digit can each be any of $0, 2, 4, 6, 8$, for $5$ choices each.
This gives $4 \cdot 5 \cdot 5 = 100$ integers for this case. \newline

If the number has three odd digits, then each of the three digits can be any of $1, 3, 5, 7$, or $9$, for
$5$ choices each. This gives $5 \cdot 5 \cdot 5 = 125$ integers for this case. \newline

In total, we have $100 + 125 = \boxed{225}$ such numbers.
\end{solution}


\begin{problem}
There are $100$ students at a math competition. Some teams consist of $5$ students while others have $6$
students. If there are a total of $18$ teams, compute the number of teams with $6$ students.
\end{problem}

\begin{answer}
$\boxed{10}$ (teams)
\end{answer}

\begin{solution}

\end{solution}


\begin{problem}%[Diego]
The STORE NAME store sells packs of $5$ highlighters for $\$3$ and packs of $6$ highlighters for $\$3.50$. 
Compute the maximum number of highlighters that may be purchased with $\$41$.
\end{problem}

\begin{answer}
$\boxed{70}$ (highlighters)
\end{answer}

\begin{solution}
We compare the two deals by comparing the unit price per highlighter for each.
The pack of $5$ highlighters has a unit price of $\frac{\$3}{5\text{ highlighters}} = \frac{36}{60}$ dollar per highlighter,
while the pack of $6$ highlighters has a unit price of $\frac{\$3.50}{6\text{ highlighters}} = \frac{35}{60}$ dollar per highlighter,
so the $6$-pack is a better deal.
Hence we should start by buying packs of $6$ highlighters. However, once we buy $10$ of these, we are left
with $\$41 - 10(\$3.50) = \$6$. At this point, we can either buy one more $6$-pack or two $5$-packs.
The latter option gives more highlighters, so we can buy a maximum of $10(6) + 2(5) = \boxed{70}$ highlighters.
\end{solution}
\end{document}
