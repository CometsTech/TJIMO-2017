\documentclass[11pt]{article}
\usepackage[paperwidth=8.5in, paperheight=11in]{geometry}

\usepackage{../tjimo}
%\usepackage[pdftex]{graphicx}
\usepackage{asymptote}
\usepackage[none]{hyphenat}

\newcommand{\sevenpoints}{Time limit: 60 minutes.}
\newcommand{\righthead}{\fdbox{Round}{Individual Solutions}}

\begin{comment}
\def\answer{\comment}
\def\solution{\comment}
\renewcommand{\righthead}{\fdbox{Round}{Individual}}
\end{comment}

\begin{document}
\setlength{\parindent}{0pt}

\begin{problem}
Compute $2^0 + 1 \cdot 7 + 2(0 \cdot 1 + 8)$.
\end{problem}

\begin{answer}
$\boxed{24}$
\end{answer}

\begin{solution}
We follow the order of operations to obtain
\begin{align*}
2^0 + 1 \cdot 7 + 2(0 \cdot 1 + 7) &= 1 + 7 + 2(0 + 8) \\
&= 8 + 2(8) \\ 
&= 8 + 16 \\
&= \boxed{24}.
\end{align*}
\end{solution}


\begin{problem}
Aaditya turned $14$ years old in the year 2017. Determine in what year Aaditya's age will be double his current age.
\end{problem}

\begin{answer}
$\boxed{2031}$
\end{answer}

\begin{solution}
Aaditya's age will be double his current age in $14$ years, or in the year $2017 + 14 = \boxed{2031}$.
\end{solution}


\begin{problem}
A fence encloses a square region with area $2025\ \text{ft}^2$. Compute the perimeter of the fence, in ft.
\end{problem}

\begin{answer}
$\boxed{180}$ (ft)
\end{answer}

\begin{solution}
The area of a square with side length $s$ is $s^2$, so the length of one side enclosed by the fence is
$\sqrt{2025\ \text{ft}^2} = 45\ \text{ft}$. There are four sides, so the perimeter is four times the side
length, which is $4 \cdot 45\ \text{ft} = \boxed{180}\ \text{ft}$.
\end{solution}


\begin{problem}
Larry is working at a job with an odd payscale. On the first day, he receives $\$1$.
On the second day, he receives $\$2$, and on the third day, he receives $3$ dollars. This continues: he receives $\$k$ on
the $k$th day, until he receives $\$28$ on the $28$th day. Calculate the total amount
of money in dollars that he receives throughout the 28 days.
\end{problem}

\begin{answer}
$\boxed{406}$ (dollars)
\end{answer}

\begin{solution}
The total number of dollars Larry receives is $1 + 2 + 3 + \cdots + 28$. We can either compute
this directly, or use a clever trick by Gauss. Observe that $1 + 28 = 2 + 27 = 3 + 26 = \cdots = 14 + 15 = 29$.
Each pair sums to $29$, and there are $14$ of these pairs, so the overall sum is $14(29) = \boxed{406}$.
\end{solution}


\begin{problem}
Paula the painter has red paint, yellow paint, green paint, and blue paint.
She wants to paint her walls with one color and the floor with a different color.
Determine the number of ways she can paint her walls and floor.
\end{problem}

\begin{answer}
$\boxed{12}$ (ways)
\end{answer}

\begin{solution}
Paula has $4$ choices for what color to paint her walls. Once she does that,
since she wants the floor to be a different color, she has $4-1 = 3$ remaining colors
to choose from. This gives a total of $4 \cdot 3 = \boxed{12}$ options.
\end{solution}


\begin{problem}
The decibel scale is used to measure the intensity of sound, or how loud a sound is.
It is designed so that a sound $10$ times as loud as another measures $10$ decibels higher.
For example, if a quiet library is at $40$ decibels, a sound $10$ times louder would be 
at $50$ decibels. If a vacuum cleaner is $1000$ times louder than the quiet library, 
compute the number of decibels for a vacuum cleaner.
\end{problem}

\begin{answer}
$\boxed{70}$ (decibels)
\end{answer}

\begin{solution}
Since the vacuum cleaner is $1000 = 10 \cdot 10 \cdot 10 = 10^3$ times louder,
that corresponds to an increase of $10 + 10 + 10 = 3(10) = 30$ decibels, so
the result is $40 + 30 = \boxed{70}$ decibels.
\end{solution}


\begin{problem}
William wants to buy a duck at The General Store. He has one coupon for $10\%$ off and another coupon for $\$10$ off.
If the price of the duck is $\$50$ and William is only allowed to use one of his two coupons,
determine the cheapest price, in dollars, at which William  can buy the duck.
\end{problem}

\begin{answer}
$\boxed{40}$ (dollars)
\end{answer}

\begin{solution}
The first coupon gives him a discount of $10\% (\$50) = 0.1(\$50) = \$5$, while the second coupon
gives him a discount of $\$10$. Clearly the second one is a better deal, so the cheapest
price at which William  can by the duck is $\$50 - \$10 = \boxed{\$40}$.
\end{solution}


\begin{problem}
Compute the number of ways Alice, Bob, Charlie, David, and Edward can stand in a line if Charlie
insists on being the person in the middle.
\end{problem}

\begin{answer}
$\boxed{24}$ (ways)
\end{answer}

\begin{solution}
Once Charlie is in the middle, the remaining $4$ people can line up in the remaining four spots
in any order, which gives $4! = 4 \cdot 3 \cdot 2 \cdot 1 = \boxed{24}$ total ways.
\end{solution}


\begin{problem}
A circular garden with area $4\pi$ has a circular walkway, shaded in the diagram below,
around the outside of the garden with width equal to the radius of the garden. 
Compute the area of the walkway (shaded region) outside the garden.
\begin{figure}[H]
\begin{center}
\begin{asy}
import graph;
unitsize(12);
filldraw(Circle((0, 0), 4), gray);
filldraw(Circle((0, 0), 2), white);

label("Garden", (1.9,0), W);
draw((3, 1) -- (6, 1));
label("Walkway", (6, 1), E);
\end{asy}
\end{center}
\end{figure}
\end{problem}

\begin{answer}
$\boxed{12\pi}$
\end{answer}

\begin{solution}
The area of a circle is given by $\pi r^2$ where $r$ is the radius of the circle.
If $\pi r^2 = 4\pi$, then $r = 2$ is the radius of the garden. Hence the width
of the garden is also $2$, so the radius of the outer circle is $2+2 = 4$.
The area of the shaded walkway equals the area of the entire outer circle,
which consists of the walkway and the garden, minus the area of the garden.
This is $\pi(4)^2 - 4\pi = 16\pi - 4\pi = \boxed{12\pi}$
\end{solution}


\begin{problem}
Today is Saturday, October 28, 2017. Determine the day of the week on which Halloween falls \textit{next} year
(October 31, 2018).
\end{problem}

\begin{answer}
$\boxed{\text{Wednesday}}$
\end{answer}

\begin{solution}
Since 2018 is not a leap year, October 28, 2018 is $365$ days after October 28, 2017. As $365 = 7(52)+1$ is
$1$ more than a multiple of $7$, October 28, 2018 will be one day after Saturday, or Sunday.
October 31 is $3$ more days later, which is $\boxed{\text{Wednesday}}$.
\end{solution}


\begin{problem}%[Kyle]
Determine how many of the following five statements in the box are false.
\begin{center}
\boxed{
\!\begin{aligned}
&\text{Exactly one of these statements is false.} \\
&\text{Exactly two of these statements are false.} \\
&\text{Exactly three of these statements are false.} \\
&\text{Exactly four of these statements are false.} \\
&\text{Exactly five of these statements are false.} 
\end{aligned}
}
\end{center}
\end{problem}

\begin{answer}
$\boxed{4}$ (statements)
\end{answer}

\begin{solution}
At most one of these statements can be true. That means four of them are false, making the
fourth statement true, and the other $\boxed{4}$ statements false.
\end{solution}


\begin{problem}%[Kyle]
In the city of Mathalopolis, there is a $\frac{1}{3}$ chance that it rains on any given day.
Compute the probability it rains in Mathalopolis at some point during two consecutive days.
\end{problem}

\begin{answer}
$\boxed{\frac{5}{9}}$
\end{answer}

\begin{solution}
The probability that it does not rain on any given day is $1 - \dfrac{1}{3} = \dfrac{2}{3}$,
so the probability it does not rain on either of two days is 
$\dfrac{2}{3} \cdot \dfrac{2}{3} = \dfrac{4}{9}$. Thus the probability that it rains at
some point between the two days is $1 - \dfrac{4}{9} = \boxed{\frac{5}{9}}$.
\end{solution}


\begin{problem}
In acute triangle $VMT$, all angles are an integer number of degrees, 
with $m\angle M = 50^\circ$. Compute the minimum possible value of $m\angle T$.
\end{problem}

\begin{answer}
$\boxed{41}$ (degrees)
\end{answer}

\begin{solution}
In order to minimize one of the angles in a triangle, we should maximize the other two angles. 
We are given $m\angle M = 50^\circ$ is fixed, and since the triangle is acute with angles of
integer degrees, the maximum possible value for $m\angle V$ is $89^\circ$.
The sum of the angles in any triangle is $180^\circ$, that means the minimum possible
value of $m\angle T$ is $180^\circ - 50^\circ - 89^\circ = \boxed{41^\circ}$.
\end{solution}


\begin{problem}
Compute the number of ordered pairs $(x, y)$ of positive even integers such that $x \cdot y = 40$.
\end{problem}

\begin{answer}
$\boxed{4}$ (pairs)
\end{answer}

\begin{solution}
Since $x$ and $y$ are positive even integers, we can write $x = 2x_1$ and $y = 2y_1$, where 
$x_1$ and $y_1$ are positive integers. Then $4x_1y_1 = 40$, or $x_1y_1 = 10$. 
At this point it is easy to see that there are $\boxed{4}$ ordered pairs of factors for $10$:
$(1, 10)$, $(2, 5)$, $(5, 2)$, and $(10, 1)$.
\end{solution}


\begin{problem}
Neeyanth has three weeks to read \textit{The Odyssey} for English class. During the first week, 
he enthusiastically reads half of the book. In the second week, he has to study for a biology test, 
so he is only able to read another $\frac{1}{6}$ of the book. Having procrastinated, 
he finishes the remaining $72$ pages of the book in the third week. Calculate the number of pages in the book.
\end{problem}

\begin{answer}
$\boxed{216}$ (pages)
\end{answer}

\begin{solution}
During the first two weeks, Neeyanth reads $\frac{1}{2} + \frac{1}{6} = \frac{2}{3}$ of the book, leaving
$1 - \frac{2}{3} = \frac{1}{3}$ of the book for the last week. We are given that he reads $72$ pages in
the third week, which is $\frac{1}{3}$ of the book, so the entire book contains $3(72) = \boxed{216}$ pages.
\end{solution}


\begin{problem}%[Kyle]
For any positive integer $n$, let $\xi(n)$ be the least common multiple of $1!, 2!, 3!$, and so
forth, up through $n!$, where $n! = n(n-1)(n-2) \cdots (2)(1)$. Compute $\frac{\xi(2018)}{\xi(2017)}$.
\end{problem}

\begin{answer}
$\boxed{2018}$
\end{answer}

\begin{solution}
Note that in general, $\xi(n) = n!$. This is because the least common multiple must be at least
the largest integer in the set, which in this case is $n!$, and clearly $n!$ is divisible
by $k!$ for all positive integers $k \le n$. Hence $\frac{\xi(2018)}{\xi(2017)} =
\frac{2018!}{2017!} = \frac{2018 \cdot 2017!}{2017!} = \boxed{2018}$.
\end{solution}


\begin{problem}
Point $A$ is at $(3, 0)$, point $B$ is at $(10, 0)$, and point $C$ is at $(0, y)$.
If the area of triangle $ABC$ is $42$, determine all possible values of $y$.
\end{problem}

\begin{answer}
$\boxed{\pm 12}$
\end{answer}

\begin{solution}
The area of a triangle is given by $\frac{1}{2} \text{base} \cdot \text{height}$.
In this case, if we let $AB = 10-3 = 7$ be the base, then the height is the perpendicular
from $C$ to $AB$, which has length $|y|$. Hence we have $\frac{1}{2}\cdot 7|y| = 42$,
or $|y| = 12$, giving $y = \boxed{\pm 12}$.
\end{solution}


\begin{problem}
There are $100$ students at a math competition. Some teams consist of $5$ students while others have $6$
students. If there are a total of $18$ teams, compute the number of teams with $6$ students.
\end{problem}

\begin{answer}
$\boxed{10}$ (teams)
\end{answer}

\begin{solution}
Suppose there are $x$ teams of $5$ students and $y$ teams of $6$ students. Then we have the
following system of equations:
\begin{align*}
\begin{cases}
5x + 6y = 100 \\
x + y = 18
\end{cases}
\end{align*}
Subtracting $5$ times the second equation from the first gives $y = 100 -  5(18) = \boxed{10}$ 
teams with $6$ students.
\end{solution}


\begin{problem}%[Kyle]
Compute the number of positive three-digit integers without leading zeros (so 042 does not count) that
have either three even digits or three odd digits.
\end{problem}

\begin{answer}
$\boxed{225}$
\end{answer}

\begin{solution}
If the number has three even digits, then the hundreds digit can be either $2, 4, 6$, or $8$, for $4$ choices,
and the tens and ones digit can each be any of $0, 2, 4, 6, 8$, for $5$ choices each.
This gives $4 \cdot 5 \cdot 5 = 100$ integers for this case. \newline

If the number has three odd digits, then each of the three digits can be any of $1, 3, 5, 7$, or $9$, for
$5$ choices each. This gives $5 \cdot 5 \cdot 5 = 125$ integers for this case. \newline

In total, we have $100 + 125 = \boxed{225}$ such numbers.
\end{solution}


\begin{problem}%[Kyle]
Harry and William are speedracers. In a race, both of them line up at the same starting point.
However, Harry knows that William runs faster than him, so William allows Harry to have
a five-second head start. If William runs $20\%$ faster than Harry and the both run at
constant speeds, compute the amount of time, in seconds, that Harry runs before William passes him.
\end{problem}

\begin{answer}
$\boxed{30}$ (seconds)
\end{answer}

\begin{solution}
Suppose Harry runs at speed $s$ meters per second (the units here are unimportant). Then William
runs at $1.2s$ meters per second. In the first five seconds, Harry gains a $5s$ meter lead.
Every second afterwards, William recovers $1.2s - s = 0.2s$ meters, so it will take
him $\frac{5}{0.2} = 25$ additional seconds to catch up. Hence Harry will run for $5 + 25 = \boxed{30}$
seconds before being passed.
\end{solution}


\begin{problem}%[Kyle]
Compute the smallest perfect square greater than $1$ that cannot be written as the sum of two
(not necessarily distinct) positive prime numbers.
\end{problem}

\begin{answer}
$\boxed{121}$
\end{answer}

\begin{solution}
It is easy to check that $4 = 2+2$, $9 = 2+7$, $16 = 5+11$, $25 = 2+23$, $36 = 7+29$, $49 = 2+47$,
$64 = 3+61$, $81 = 2+79$, and $100 = 3+97$. Note that an odd perfect square can only be the
sum of an even and odd prime, and the only even prime is $2$. Since $11^2 - 2 = 119$ is not prime,
it is impossible to write $11^2 = \boxed{121}$ as the sum of two positive primes.
For small even squares, it is not difficult to simply test out cases. The more general problem
is known as Goldbach's conjecture.
\end{solution}


\begin{problem}%[Kyle]
A rectangle is inscribed inside a circle with area $9\pi$. From each side of the rectangle,
a square is extruded outward, as shown by the dashed lines in the diagram below. 
Compute the sum of the areas of the four squares.
\begin{figure}[H]
\begin{center}
\begin{asy}
import graph;
unitsize(6);
draw(Circle((0,0), 5));
draw((3,4) -- (3,-4) -- (-3,-4) -- (-3,4)--cycle);
draw((3,4) -- (3,10) -- (-3,10) -- (-3,4), dashed);
draw((3,4) -- (11,4) -- (11,-4) -- (3,-4), dashed);
draw((-3,-4) -- (-3,-10) -- (3,-10) -- (3,-4), dashed);
draw((-3,-4) -- (-11,-4) -- (-11,4) -- (-3,4), dashed);
\end{asy}
\end{center}
\end{figure}
\end{problem}

\begin{answer}
$\boxed{72}$
\end{answer}

\begin{solution}
Let the length and width of the rectangle be $l$ and $w$, respectively. Then the sum of 
the areas of the four squares is $2(l^2 + w^2)$. By the Pythagorean Theorem,
$l^2 + w^2 = d^2$, where $d$ is the length of the diagonal, which is also the diameter
of the circle. The area of a circle is given by $\pi r^2 = \frac{1}{4}\pi d^2 = 9\pi$,
so $d^2 = 36$. Hence $2(l^2 + w^2) = 2(36) = \boxed{72}$.
\end{solution}


\begin{problem}%[Kyle]
Compute the number of positive integers $N$ such that $5 \cdot N$ has two digits while $6 \cdot N$ has three digits.
\end{problem}

\begin{answer}
$\boxed{3}$ (numbers)
\end{answer}

\begin{solution}
We will look at the two conditions separately and see which numbers satisfy both. 
Define $\lfloor x \rfloor$ to be the greatest integer less than or equal to $x$ and $\lceil x \rceil$ to be the smallest integer greater than or equal to $x$.
If $5 \cdot N$ has two digits,
then $N$ must be between $\frac{10}{5} = 2$ and $\left\lfloor \frac{99}{5} \right\rfloor = 19$, inclusive.
If $6 \cdot N$ has three digits, then $N$ must be between $\left\lceil \frac{100}{6} \right\rceil = 17$ and
$\left\lfloor \frac{999}{6} \right\rfloor = 166$, inclusive. There are $\boxed{3}$ numbers in the intersection
of these sets: $17$, $18$, and $19$.
\end{solution}


\begin{problem}%[Diego]
The General Store sells packs of $5$ highlighters for $\$3$ and packs of $6$ highlighters for $\$3.50$. 
Compute the maximum number of highlighters that may be purchased with $\$41$.
\end{problem}

\begin{answer}
$\boxed{70}$ (highlighters)
\end{answer}

\begin{solution}
We compare the two deals by comparing the unit price per highlighter for each.
The pack of $5$ highlighters has a unit price of $\frac{\$3}{5\text{ highlighters}} = \frac{36}{60}$ dollar per highlighter,
while the pack of $6$ highlighters has a unit price of $\frac{\$3.50}{6\text{ highlighters}} = \frac{35}{60}$ dollar per highlighter,
so the $6$-pack is a better deal.
Hence we should start by buying packs of $6$ highlighters. However, once we buy $10$ of these, we are left
with $\$41 - 10(\$3.50) = \$6$. At this point, we can either buy one more $6$-pack or two $5$-packs.
The latter option gives more highlighters, so we can buy a maximum of $10(6) + 2(5) = \boxed{70}$ highlighters.
\end{solution}


\begin{problem}%[Kyle]
One percent of all mathletes are legendary. Some mathletes are oracles, whose job is to predict
whether a given mathlete is legendary. Oracles make correct predictions $95\%$ of the time.
Kural the mathlete asks an oracle whether he is legendary, and the oracle tells him that he is.
Determine the probability that he is actually legendary.
\end{problem}

\begin{answer}
$\boxed{\frac{19}{118}}$
\end{answer}

\begin{solution}
This is a problem in conditional probability. There are two possible results in our sample space.
Either Kural is legendary and the oracle predicts correctly, which occurs with probability
$\frac{1}{100} \cdot \frac{19}{20}$, or Kural is not legendary and the oracle predicts incorrectly, 
which occurs with probability $\frac{99}{100} \cdot \frac{1}{20}$. The former is the result we
want, so the probability is $\dfrac{\frac{1}{100} \cdot \frac{19}{20}}{\frac{1}{100} \cdot \frac{19}{20} + \frac{99}{100} \cdot \frac{1}{20}}
= \dfrac{1 \cdot 19}{1 \cdot 19 + 99 \cdot 1} = \boxed{\frac{19}{118}}$.
\end{solution}


\begin{problem}%[Kyle]
Convex quadrilateral $TJHS$ has side lengths $TJ = JH = 5$, $HS = 1$, and $ST = 7$. If $\angle TJH$ is a right
angle, compute the area of quadrilateral $TJHS$.
\end{problem}

\begin{answer}
$\boxed{16}$
\end{answer}

\begin{solution}
We are given that triangle $TJH$ is a right triangle, so by the Pythagorean Theorem, 
$TH^2 = TJ^2 + JH^2 = 5^2 + 5^2 = 50$. However, we also have $HS^2 + ST^2 = 1^2 + 7^2 = 50 = TH^2$, 
so by the converse, triangle $HST$ is also a right triangle with $\angle HST$ being a right angle. 
The area of the quadrilateral is then sum of the area of the two right triangles, 
which is $\frac{1}{2}(5)(5) + \frac{1}{2}(1)(7) = \frac{25}{2} + \frac{7}{2} = \boxed{16}$.
\end{solution}


\begin{problem}%[Kyle]
Compute the positive real number $x$ such that $\sqrt[3]{x + 3\sqrt[3]{x + 3\sqrt[3]{x + \cdots}}} = 3$.
\end{problem}

\begin{answer}
$\boxed{18}$
\end{answer}

\begin{solution}
Cubing both sides of the equation gives $x + 3\sqrt[3]{x+3\sqrt[3]{x + \cdots}} = 27$.
However, we already know that $\sqrt[3]{x + 3\sqrt[3]{x + 3\sqrt[3]{x + \cdots}}} = 3$, and this expression 
is repeated in the second term on the left side, so the equation is equivalent to $x + 3(3) = 27$.
Subtracting $3(3) = 9$ from both sides gives $x = \boxed{18}$.
\end{solution}


\begin{problem}
Find all possible areas of rectangles with integer side lengths whose area equals its perimeter.
\end{problem}

\begin{answer}
$\boxed{16, 18}$
\end{answer}

\begin{solution}
Let $l$ and $w$ be the length and width of the rectangle, respectively. Its perimeter is
$2l + 2w$ and its area is $lw$, so $lw = 2w + 2l$, or $lw - 2w - 2l + 4 = 4$. Thus
$(l-2)(w-2) = 4$. Since $l$ and $w$ are integers, we look at possible integer factors
of $4$: $1 \cdot 4$ or $2 \cdot 2$. If $l-2$ and $w-2$ are $1$ and $4$, in some order,
then $l$ and $w$ are $3$ and $6$, which gives an area of $\boxed{18}$. If $l-2$
and $w-2$ are $2$ and $2$, then $l$ and $w$ are $4$ and $4$, which gives an area of $\boxed{16}$.
\end{solution}


\begin{problem}%[Kevin Zhang]
Compute the last four digits in $101^{2017}$.
\end{problem}

\begin{answer}
$\boxed{1701}$
\end{answer}

\begin{solution}
We start by expanding $101^{2017} = (1 + 100)^{2017}$ with the Binomial Theorem:
\[(1 + 100)^{2017} = 1 + \binom{2017}{1}(100) + \binom{2017}{2}(100)^2 + 
\binom{2017}{3}(100)^3 + \cdots + \binom{2017}{2016}100^{2016} + 100^{2017}.\]
Note that $100^n$ ends in four zeros for all $n \ge 2$. Hence it suffices to look at the first two terms in the 
expansion: $1 + \binom{2017}{1}(100) = 1 + 2017(100) = 201701$. The last
four digits are $\boxed{1701}$.
\end{solution}


\begin{problem}
Quadrilateral $ABCD$ has sides $AB = 1009$, $BC = 2018$, and $CD = 2018$. If $\angle BAD$ is a right angle 
and $m\angle ADC = \frac{1}{2}m\angle BCD$, compute $m\angle BCD$ in degrees.
\end{problem}

\begin{answer}
$\boxed{108}$ (degrees)
\end{answer}

\begin{solution}
Reflect the quadrilateral about side $AD$, and let $B'$ be the reflection of $B$ and $C'$ that of $C$. 
Then pentagon $BCDC'B'$ is equilateral. Furthermore, $\angle BCD \cong \angle CDC' \cong DC'B'$, 
and this construction is uniquely defined, so the pentagon is in fact regular. 
Therefore $\angle BCD$ is an angle in a regular pentagon, so it measures $\boxed{108^\circ}$.
\end{solution}
\end{document}
