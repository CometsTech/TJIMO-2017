\documentclass[11pt]{article}
\usepackage[paperwidth=8.5in, paperheight=11in]{geometry}

\usepackage{../tjimo}
%\usepackage[pdftex]{graphicx}
\usepackage{asymptote}
\usepackage[none]{hyphenat}

\begin{comment}
\def\answer{\comment}
\def\solution{\comment}
\def\solutionone{\comment}
\def\solutiontwo{\comment}
\end{comment}

\newcommand{\sevenpoints}{Time limit: 60 minutes.}
\newcommand{\righthead}{\fdbox{Round}{Individual}}

\begin{document}
\setlength{\parindent}{0pt}

\begin{problem}
Compute $2^0 + 1 \cdot 7 + 2(0 \cdot 1 + 8)$.
\end{problem}

\begin{answer}
$\boxed{24}$
\end{answer}

\begin{solution}
We follow the order of operations to obtain
\begin{align*}
2^0 + 1 \cdot 7 + 2(0 \cdot 1 + 7) &= 1 + 7 + 2(0 + 8) \\
&= 8 + 2(7) \\ 
&= 8 + 16 \\
&= \boxed{24}.
\end{align*}
\end{solution}


\begin{problem}
Aaditya turned $14$ years old in the year 2017. Determine in what year Aaditya's age will double.
\end{problem}

\begin{answer}
$\boxed{2031}$
\end{answer}

\begin{solution}
Aaditya's age will double in $14$ years, or in the year $2017 + 14 = \boxed{2031}$.
\end{solution}


\begin{problem}
A fence encloses a square region with area $2025\ \text{ft}^2$. Compute the perimeter of the fence, in ft.
\end{problem}

\begin{answer}
$\boxed{180}$ (ft)
\end{answer}

\begin{solution}
The area of a square with side length $s$ is $s^2$, so the length of one side enclosed by the fence is
$\sqrt{2025\ \text{ft}^2} = 45\ \text{ft}$. There are four sides, so the perimeter is four times the side
length, which is $4 \cdot 45\ \text{ft} = \boxed{180}\ \text{ft}$.
\end{solution}


\begin{problem}
Compute the number of ways Alice, Bob, Charlie, David, and Edward can stand in a line if Charlie
insists on being the person in the middle.
\end{problem}

\begin{answer}
$\boxed{24}$ (ways)
\end{answer}

\begin{solution}
Once Charlie is in the middle, the remaining $4$ people can line up in the remaining four spots
in any order, which gives $4! = 4 \cdot 3 \cdot 2 \cdot 1 = \boxed{24}$ total ways.
\end{solution}


\begin{problem}
Today is Saturday, October 28, 2017. Determine the day of the week on which Halloween falls \textit{next} year
(October 31, 2018).
\end{problem}

\begin{answer}
$\boxed{\text{Wednesday}}$
\end{answer}

\begin{solution}
Since 2018 is not a leap year, October 28, 2018 is $365$ days after October 28, 2017. As $365 = 7(52)+1$ is
$1$ more than a multiple of $7$, October 28, 2018 will be one day after Saturday, or Sunday.
October 31 is $3$ more days later, which is $\boxed{\text{Wednesday}}$.
\end{solution}


\begin{problem}%[Kyle]
Determine how many of the following five statements are false.
\begin{itemize}
\item Exactly one of these statements is false.
\item Exactly two of these statements is false.
\item Exactly three of these statements is false.
\item Exactly four of these statements is false.
\item Exactly five of these statements is false.
\end{itemize}
\end{problem}

\begin{answer}
$\boxed{4}$ (statements)
\end{answer}

\begin{solution}
At most one of these statements can be true. That means four of them are false, making the
fourth statement true, and the other $\boxed{4}$ statements false.
\end{solution}


\begin{problem}%[Kyle]
In the city of Mathalopolis, there is a $\frac{1}{3}$ chance that it rains on any given day.
What isthe probability it rains in Mathalopolis at some point during two consecutive days?
\end{problem}

\begin{answer}
$\boxed{\frac{5}{9}}$
\end{answer}

\begin{solution}
The probaility that it does not rain on any given day is $1 - \dfrac{1}{3} = \dfrac{2}{3}$,
so the probability it does not rain on either of two days is 
$\dfrac{2}{3} \cdot \dfrac{2}{3} = \dfrac{4}{9}$. Thus the probability that it rains at
some point between the two days is $1 - \dfrac{4}{9} = \boxed{\frac{5}{9}}$.
\end{solution}


\begin{problem}
In acute triangle $VMT$, all angles are an integer number of degrees, 
with $m\angle M = 50^\circ$. Compute the minimum possible value of $m\angle T$.
\end{problem}

\begin{answer}
$\boxed{41}$ (degrees)
\end{answer}

\begin{solution}
In order to minimize one of the angles in a triangle, we should maximize the other two angles. 
We are given $m\angle M = 50^\circ$ is fixed, and since the triangle is acute with angles of
integer degrees, the maximum possible value for $m\angle V$ is $89^\circ$.
The sum of the angles in any triangle is $180^\circ$, that means the minimum possible
value of $m\angle T$ is $180^\circ - 50^\circ - 89^\circ = \boxed{41^\circ}$.
\end{solution}


\begin{problem}
NAME has three weeks to read BOOK for English class. During the first week, 
enthusiastic NAME reads half of the book. In the second week, NAME has to study for a biology test, 
so he is only able to read another $\frac{1}{6}$ of the book. Having procrastinated, 
NAME finishes the remaining $65$ pages of the book in the third week. Calculate the number of pages in the book.
\end{problem}

\begin{answer}
$\boxed{195}$ (page)
\end{answer}

\begin{solution}
During the first two weeks, NAME reads $\frac{1}{2} + \frac{1}{6} = \frac{2}{3}$ of the book, leaving
$1 - \frac{2}{3} = \frac{1}{3}$ of the book for the last week. We are given that NAME reads $65$ pages in
the third week, which is $\frac{1}{3}$ of the book, so the entire book contains $3(65) = \boxed{195}$ pages.
\end{solution}


\begin{problem}%[Kyle]
Compute the smallest perfect square greater than $1$ that cannot be written as the sum of two
(not necessarily distinct) positve prime numbers.
\end{problem}

\begin{answer}
$\boxed{121}$
\end{answer}

\begin{solution}
It is easy to check that $4 = 2+2$, $9 = 2+7$, $16 = 5+11$, $25 = 2+23$, $36 = 7+29$, $49 = 2+47$,
$64 = 3+61$, $81 = 2+79$, and $100 = 3+97$. Note that an odd perfect square can only be the
sum of an even and odd prime, and the only even prime is $2$. Since $11^2 - 2 = 119$ is not prime,
it is impossible to write $11^2 = \boxed{121}$ as the sum of two positive primes.
For small even squares, it is not difficult to simply test out cases. The more general problem
is known as Goldbach's conjecture.
\end{solution}


\begin{problem}
Compute the number of positive three-digit integers without leading zeros (so 042 does not count) that
have either three even digits or three odd digits.
\end{problem}

\begin{answer}
$\boxed{225}$
\end{answer}

\begin{solution}%[Kyle]
If the number has three even digits, then the hundreds digit can be either $2, 4, 6$, or $8$, for $4$ choices,
and the tens and ones digit can each be any of $0, 2, 4, 6, 8$, for $5$ choices each.
This gives $4 \cdot 5 \cdot 5 = 100$ integers for this case. \newline

If the number has three odd digits, then each of the three digits can be any of $1, 3, 5, 7$, or $9$, for
$5$ choices each. This gives $5 \cdot 5 \cdot 5 = 125$ integers for this case. \newline

In total, we have $100 + 125 = \boxed{225}$ such numbers.
\end{solution}


\begin{problem}
There are $100$ students at a math competition. Some teams consist of $5$ students while others have $6$
students. If there are a total of $18$ teams, compute the number of teams with $6$ students.
\end{problem}

\begin{answer}
$\boxed{10}$ (teams)
\end{answer}

\begin{solution}
Suppose there are $x$ teams of $5$ students and $y$ teams of $6$ students. Then we have the
following system of equations:
\begin{align*}
\begin{cases}
5x + 6y = 100 \\
x + y = 18
\end{cases}
\end{align*}
Subtracting $5$ times the second equation from the first gives $y = 100 -  5(18) = \boxed{10}$ 
teams with $6$ students.
\end{solution}


\begin{problem}%[Kyle]
Compute the number of positive integers $N$ such that $5 \cdot N$ has two digits while $6 \cdot N$ has three digits.
\end{problem}

\begin{answer}
$\boxed{3}$ (numbers)
\end{answer}

\begin{solution}
We will look at the two conditions separately and se which numbers satisfy both. If $t \cdot N$ has two digits,
then $N$ must be between $\frac{10}{5} = 2$ and $\left\lfloor \frac{99}{5} \right\rfloor = 19$, inclusive.
If $6 \cdot N$ has three digits, then $N$ must be between $\left\lceil \frac{100}{6} \right\rceil = 17$ and
$\left\lfloor \frac{999}{6} \right\rceil = 166$, inclusive. There are $\boxed{3}$ numbers in the intersection
of these sets: $17$, $18$, and $19$.
\end{solution}


\begin{problem}%[Diego]
The General Store sells packs of $5$ highlighters for $\$3$ and packs of $6$ highlighters for $\$3.50$. 
Compute the maximum number of highlighters that may be purchased with $\$41$.
\end{problem}

\begin{answer}
$\boxed{70}$ (highlighters)
\end{answer}

\begin{solution}
We compare the two deals by comparing the unit price per highlighter for each.
The pack of $5$ highlighters has a unit price of $\frac{\$3}{5\text{ highlighters}} = \frac{36}{60}$ dollar per highlighter,
while the pack of $6$ highlighters has a unit price of $\frac{\$3.50}{6\text{ highlighters}} = \frac{35}{60}$ dollar per highlighter,
so the $6$-pack is a better deal.
Hence we should start by buying packs of $6$ highlighters. However, once we buy $10$ of these, we are left
with $\$41 - 10(\$3.50) = \$6$. At this point, we can either buy one more $6$-pack or two $5$-packs.
The latter option gives more highlighters, so we can buy a maximum of $10(6) + 2(5) = \boxed{70}$ highlighters.
\end{solution}


\begin{problem}
Convex quadrilateral $TJHS$ has side lengths $TJ = JH = 5$, $HS = 1$, and $ST = 7$. If $\angle TJH$ is a right
angle, compute the area of quadrilateral $TJHS$.
\end{problem}

\begin{answer}
$\boxed{16}$
\end{answer}

\begin{solution}
We are given that triangle $TJH$ is a right triangle, so by the Pythagorean Theorem, 
$TH^2 = TJ^2 + JH^2 = 5^2 + 5^2 = 50$. However, we also have $HS^2 + ST^2 = 1^2 + 7^2 = 50 = TH^2$, 
so by the converse, triangle $HST$ is also a right triangle with $\angle HST$ being a right angle. 
The area of the quadrilateral is then sum of the area of the two right triangles, 
which is $\frac{1}{2}(5)(5) + \frac{1}{2}(1)(7) = \frac{25}{2} + \frac{7}{2} = \boxed{16}$.
\end{solution}


\begin{problem}
Quadrilateral $ABCD$ has sides $AB = 1009$, $BC = 2018$, and $CD = 2018$. If $\angle BAD$ is a right angle 
and $m\angle ADC = \frac{1}{2}m\angle BCD$, compute $m\angle BCD$ in degrees.
\end{problem}

\begin{answer}
$\boxed{108}$ (degrees)
\end{answer}

\begin{solution}
Reflect the quadrilateral about side $AD$, and let $B'$ be the reflection of $B$ and $C'$ that of $C$. 
Then pentagon $BCDC'D'$ is equilateral. Furthermore, $\angle BCD \cong \angle CDC' \cong DC'B'$, 
and this construction is uniquely defined. Therefore $\angle BCD$ is an angle in a regular pentagon, 
so it measures $\boxed{108^\circ}$.
\end{solution}
\end{document}
