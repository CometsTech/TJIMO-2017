\documentclass[11pt]{article}
%\usepackage[paperwidth=8.5in, paperheight=11in]{geometry}
\usepackage[a5paper, landscape]{geometry}
\usepackage{tikz}

\usepackage{../tjimo}
%\usepackage[pdftex]{graphicx}

\begin{comment}
\def\answer{\comment}
\def\solution{\comment}
\def\solutionone{\comment}
\def\solutiontwo{\comment}
\end{comment}

%\newcommand{\sevenpoints}{Time limit: 40 minutes.}
\newcommand{\righthead}{\fdbox{Round}{Practice Guts}}

\begin{document}

\section*{Set 1}
\begin{problem}
What is $1-2+3-4+5-6+7-8 + \dots + 19-20$?
\end{problem}

\begin{answer}
-10
\end{answer}

\begin{solution}
Note that $1-2=3-4=5-6=\dots=19-20=-1$. Since there are 10 of these such terms, we get that the answer is $10\cdot -1 = \boxed{-10}$.
\end{solution}

\begin{problem}
What is the area of a triangle with side lengths 13, 14, and 15?
\end{problem}

\begin{answer}
84
\end{answer}

\begin{solution}
Note that if we drop an altitude to the side with length 14, we form a 5-12-13 right triangle and a 3-4-5 right triangle. The altitude has a height of 12. Thus, the area is $\frac{1}{2}(12)(14) = \boxed{84}$.
\end{solution}

\begin{problem}
How many numbers are between 505 and 700, inclusive?
\end{problem}

\begin{answer}
196
\end{answer}

\begin{solution}
If we subtract 504 from each of the numbers, we get all the numbers from 1 to 196, inclusive. There are 196 of these numbers, so there must have been $\boxed{196}$ numbers in the original set.
\end{solution}

\begin{problem}
What is the greatest common factor of 117 and 156?
\end{problem}

\begin{answer}
39
\end{answer}

\begin{solution}
The greatest common factor of 117 and 156 is the same as the greatest common factor between 117 and $156-117=39$. Since $117=39(3)$, the greatest common factor must be $\boxed{39}$
\end{solution}
\newpage
\section*{Set 2}
\begin{problem}
It takes Marbury 1 hour to deliver 6 letters and Madison 3 hours to deliver 60 letters. How many letters can they deliver in an 8 hour work day?
\end{problem}

\begin{answer}
208
\end{answer}

\begin{solution}
Marbury can deliver $6(8)=48$ letters in 8 hours, and Madison can deliver $\frac{60}{3}(8)=160$ letters in 8 hours. Summing these up yields $\boxed{208}$ letters in an 8 hour work day.
\end{solution}

\begin{problem}
In quadrilateral $ABCD$, $\angle DAC = 75^\circ$, $\angle ACB = 40^\circ$, $\angle DBC = 75^\circ$, and $\angle BDC = 25^\circ$. Find the measure of angle $\angle DCA$.
\end{problem}

\begin{answer}
$40^\circ$
\end{answer}

\begin{solution}
We observe that $\angle DCA = 180^\circ - \angle DBC - \angle ACB - \angle BDC = 40^\circ$.
\end{solution}

\begin{problem}
What are the sum of the factors of 16?
\end{problem}

\begin{answer}
31
\end{answer}

\begin{solution}
Note that the factors of $16=2^4$ are 1, 2, 4, 8, and 16. Summing these yields $31$.
\end{solution}

\begin{problem}
There are 100 people in math team. If 53 of them do cross country, 27 of them do art club, and 38 of them do neither, how many do both?
\end{problem}

\begin{answer}
18
\end{answer}

\begin{solution}
If 38 do neither, then $100-38=62$ must do either art club or cross country. Since $53$ do cross country, only $62-53=9$ must do only art club, leaving $27-9=\boxed{18}$ that do both.
\end{solution}

\newpage
\section*{Set 3}
\begin{problem}
In a round robin tournament, everyone competes against everyone else. If there are 8 teams, how many matches are there?
\end{problem}

\begin{answer}
28
\end{answer}

\begin{solution}
If every team plays against everyone else, each team plays against 7 other teams. Since each match has 2 teams playing simultaneously, we have a total of $\frac{(8)(7)}{2}=\boxed{28}$ total matches.
\end{solution}

\begin{problem}
If I roll three die, what is the probability the numbers on the three die sum to 16?
\end{problem}

\begin{answer}
$\frac{1}{36}$
\end{answer}

\begin{solution}
The only ways to sum to $16$ are if I roll $5, 5, 6$, or a $4, 6, 6$. These each have a probability of $\frac{3}{216}$ of occurring. Summing this up, we have the probability that they sum to 16 is equal to $\frac{6}{216} = \boxed{\frac{1}{36}}$
\end{solution}

\begin{problem}
I have a rectangle with perimeter 36. What is the maximum possible area of the rectangle?
\end{problem}

\begin{answer}
81
\end{answer}

\begin{solution}
Let the side lengths of the rectangle be $x$ and $y$. Since the perimeter is 36, we see that $x+y$ is 18. Using the AM-GM inequality, we see that $\dfrac{x+y}{2} \geq \sqrt{xy}$. Thus, the maximum possible value of $xy$, or the area of the rectangle, is $\left(\dfrac{18}{2}\right)^2=\boxed{81}$ 
\end{solution}

\begin{problem}
Let $20ABC16$ be a perfect square, with $A$, $B$, and $C$ as digits. What is the three-digit number $ABC$?
\end{problem}

\begin{answer}
909
\end{answer}

\begin{solution}    
Note that $\sqrt{2000000} = 1000\sqrt{2}$, or about 1414. Let $20ABC16$ be equal to $x^2$. Now, consider perfect squares $\mod$ 100. They must be equal to $16 \mod 4$ and $16 \mod 25$. This leads us to the conclusion that $x$ is either $0$  or $2 \mod 4$, and either $4$ or $21$ mod 25. Thus, the last two digits of $x$ must be either $04, 46, 54,$ or $96$. Note that 1404 is too small, and trying 1446 yields $2090916$, giving us our desired answer of $\boxed{909}$.
\end{solution}

\end{document}
