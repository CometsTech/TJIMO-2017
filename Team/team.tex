\documentclass[11pt]{article}
\usepackage[paperwidth=8.5in, paperheight=11in]{geometry}

\usepackage{../tjimo}
%\usepackage[pdftex]{graphicx}
\usepackage{asymptote}
\usepackage[none]{hyphenat}

\newcommand{\sevenpoints}{Time limit: 30 minutes.}
\newcommand{\righthead}{\fdbox{Round}{Team Solutions}}

%\begin{comment}
\def\answer{\comment}
\def\solution{\comment}
\renewcommand{\righthead}{\fdbox{Round}{Team}}
%\end{comment}

\begin{document}
\setlength{\parindent}{0pt}

\begin{problem}
There are $200$ animals on a farm. Some are chickens and some are horses. If there are $450$ legs in total, how many horses are there?
(Chickens have two legs each, and horses have four legs each.)
\end{problem}

\begin{answer}
$25$
\end{answer}

\begin{solution}
Let $c$ be the number of chickens, and $h$ be the number of horses. We have the following system of equations:
\begin{equation}
c+h = 200 \\
\end{equation}
\begin{equation}
2c + 4h = 450
\end{equation}
We can multiply the first equation by 2 to get $2c + 2h = 200$. Since $2c + 4h = 250$, we must have $2h = 50$, or $h = \boxed{25}$.
\end{solution}


\begin{problem}
There are $n$ students in a class. The teacher can divide them evenly into groups of $2$, $3$, $4$, $5$, and $7$. If $1000 < n < 1500$, find $n$.
\end{problem}

\begin{answer}
$1260$
\end{answer}

\begin{solution}
We first want to find the least possible number $n$ that divides $2$, $3$, $4$, $5$, and $7$. 
We know that any multiple of $4$ is automatically a multiple of $2$, so we only need to require $n$ to be divisible by $3$, $4$, $5$, and $7$. 
Since $3,4,5,7$ share no common factors, any such $n$ must be divisible by $3 \cdot 4 \cdot 5 \cdot 7 = 420$. 
The only multiple of $420$ in our given range is $420 \cdot 3 = \boxed{1260}$.
\end{solution}


\begin{problem}
An integer is chosen randomly from the first $2017^2$ positive integers. What is the probability that the chosen integer is a perfect square?
\end{problem}
\begin{answer}
$\dfrac{1}{2017}$
\end{answer}
\begin{solution}
Note that the squares between $1$ and $2017^2$ are \[1^2,\ 2^2,\ 3^2,\ \cdots,\ 2016^2,\ 2017^2.\]
There are $2017$ of these squares out of a total $2017^2$ positive integers. 
This gives a probability of $\dfrac{2017}{2017^2} = \boxed{\dfrac{1}{2017}}$.
\end{solution}


\begin{problem}
What is the smallest positive integer that is 
\begin{itemize}
    \item greater than $1$
    \item not prime
    \item not divisible by $2$, $3$, or $5$?
\end{itemize}
\end{problem}
\begin{answer}
$49$
\end{answer}
\begin{solution}
Any positive integer that is greater than $1$ and not prime must be composite. 
That means our answer must be the product of two integers greater than $1$. Let our answer be $ab$, with $a,b > 1$. 
We notice that if either $a$ or $b$ is one of $2, 3, 4, 5, 6$, then the product would be divisible by $2$, $3$, or $5$, which is not allowed. 
Thus, $a, b \ge 7$ and our smallest possible integer satisfying the conditions is $7 \cdot 7 = \boxed{49}$.
\end{solution}


\begin{problem}
Point $C$ is the center of the rectangle shown below. If the area of the whole rectangle is $80$, find the area of the shaded region.
\begin{center}
\begin{asy}
size(100);
draw((0,0)--(3,0)--(3,5)--(0,5)--(0,0));
draw((3,0)--(1.5, 2.5));
draw((0,0)--(1.5, 2.5));
dot((1.5, 2.5));
label("C", (1.5, 2.5), NE);
fill((0,0)--(1.5,2.5)--(3,0)--(0,0)--cycle,gray);
\end{asy}
\end{center}
\end{problem}
\begin{answer}
$20$
\end{answer}
\begin{solution}
Extend the lines in the diagram as such: 
\begin{center}
\begin{asy}
size(100);
draw((0,0)--(3,0)--(3,5)--(0,5)--(0,0));
fill((0,0)--(1.5,2.5)--(3,0)--(0,0)--cycle,gray);
draw((3,0)--(0, 5));
draw((0,0)--(3, 5));
draw((0,2.5)--(3, 2.5));
draw((1.5,0)--(1.5, 5));
dot((1.5, 2.5));
label("C", (1.5, 2.5), NE);
\end{asy}
\end{center}
We divided the rectangle into eight congruent triangles, two of which are shaded. 
Thus $\dfrac{1}{4}$ of the total area is shaded, so our answer is $80 \cdot \dfrac14 = \boxed{20}.$
\end{solution}


\begin{problem}
A four-digit positive integer $\overline{ABCD}$ is called $\textit{funny}$ if each of its digits is between $1$ and $9$, and both $\overline{ABC}$ and $\overline{BCD}$ are perfect squares. For example, $1441$ is $\textit{funny}$ because $144 = 12^2$ and $441 = 21^2$, but $9009$ is not $\textit{funny}$ because it has zeros as digits. What is the largest four-digit $\textit{funny}$ integer?
\end{problem}

\begin{answer}
$7841$
\end{answer}

\begin{solution}
We will check all possible three-digit squares $\overline{ABC}$ in descending order, and for each square we will check whether the last two digits could be the first two digits in another three-digit square. $32^2 = 961$ does not work because there is no three digit square between $611$ and $619$. $30^2 = 900$ has zeros, so it cannot be part of a $\textit{funny}$ integer. $29^2 = 841$ does not work because there is no square between $411$ and $419$. Finally, $28^2 = 784$ works because $29^2 = 841$ starts with the digits ``84". Our largest $\textit{funny}$ integer is therefore $\boxed{7841}$.
\end{solution}


\begin{problem}
How many positive two-digit integers are $\dfrac{4}{7}$ of their ``reverse"? (For example, $48$ is $\dfrac{4}{7}$ of $84$, the number formed by swapping the digits of $48$.)
\end{problem}

\begin{answer}
$4$
\end{answer}

\begin{solution}
Let one such two-digit number be $\overline{AB}$. We are given $7*\overline{AB} = 4*\overline{BA}$. Notice that we can write $\overline{AB} = 10A+B$ and $\overline{BA} = 10B + A$. This gives us $7(10A + B) = 4(10B + A)$. Expanding and simplifying, we get $66A = 33B$, or $2A = B$. This gives us the solutions $(A, B) = (1, 2), (2, 4), (3, 6), (4, 8)$. Notice we omit the pairs $(0, 0)$ and $(5, 10)$ because they would not result in two-digit numbers. Our four ordered pairs correspond to the two digit numbers $12, 24, 36, 48$. Thus, there are $\boxed{4}$ such two-digit numbers. 
\end{solution}


\begin{problem}
Suppose $x,y$ are positive real numbers such that \[ xy + x^2 = 23 \]  \[ xy + y^2 = 26.\] Compute $x+y$.
\end{problem}

\begin{answer}
$7$
\end{answer}

\begin{solution}
Summing these two equations yields $x^2 + 2xy + y^2 = 49$, so $(x+y)^2 = 49$, and since $x$ and $y$ are positive, we conclude $x+y = \boxed{7}$.
\end{solution}


\begin{problem}
Dan the Doubler has an interesting way of doubling numbers. He takes each digit of a number, doubles it, and then puts the answers together. 
For example, he ``doubles" $666$ to get $121212$ and ``doubles" $202$ to get $404$. For how many three-digit integers from $100$ to $999$,
inclusive, does Dan get the correct answer using his doubling method? (For example, $202 \times 2 = 404$ is correct, but $666 \times 2 = 121212$ is incorrect.)
\end{problem}

\begin{answer}
225
\end{answer}

\begin{solution}
Notice that Dan the Doubler's doubling method works if and only if there are no carries necessary when adding the number to itself. The digits $\{0, 1, 2, 3, 4\}$ are the only digits which do not require a carry when added to themselves. Thus, the tens and ones digit of the three digit number must be one of these five digits. The hundreds digit, however, can be any non-zero digit. (To understand why, notice how when doubling $944$ or $833$, no carries are necessary.) Since there are $5$ possibilities for each of the tens and ones digits, and $9$ for the hundreds digit, our answer is $5 \cdot 5 \cdot 9 = \boxed{225}$.
\end{solution}


\begin{problem}
The product of the lengths of the three altitudes of a certain triangle is 24. If this triangle has an area of 3, then determine the product of the three side lengths of this triangle.  (The altitudes of a triangle are the line segments connecting the vertices and the feet of the perpendiculars from those vertices to the opposite sides.) % KYLE
\end{problem}

\begin{answer}
$9$
\end{answer}

\begin{solution}
Let $a$, $b$, $c$ be the length of the sides of the triangle, and $h_a$, $h_b$, $h_c$ be the lengths of the altitudes of sides $a$, $b$, and $c$ respectively. We know $h_a \cdot h_b \cdot h_c = 24$. We also know that the area of the triangle is $3 = \dfrac12 \cdot a \cdot h_a = \dfrac12 \cdot b \cdot h_b = \dfrac12 \cdot c \cdot h_c$, so $ah_a = bh_b = ch_c = 6$. Thus, $(ah_a)(bh_b)(ch_c) = abc \cdot h_ah_bh_c = 6^3 = 216$, so $abc = \dfrac{216}{h_ah_bh_c} = \dfrac{216}{24} = \boxed{9}$.
\end{solution}
\end{document}

