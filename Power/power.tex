\documentclass[11pt]{article}
\usepackage[paperwidth=8.5in, paperheight=11in]{geometry}

\usepackage{subfig}
\usepackage{../tjimo}

\newcommand{\sevenpoints}{Time limit: 45 minutes.}
\newcommand{\righthead}{\fdbox{Round}{Power}}

\begin{comment}
\def\answer{\comment}
\def\solution{\comment}
\end{comment}

\renewenvironment{problem}{\begin{problems}}{\end{problems}\vspace{5pt}}

\begin{document}

%\setlength{\belowcaptionskip}{-50pt}
\setlength{\parindent}{0pt}

\section{Introduction}

Unlike the other rounds, just getting the answer right is not enough on the Power Round. 
Make sure you explain your answer and use words to describe how you arrived at your answer. 
In the words of middle school math teachers across the nation -- no work, no credit! \newline

Feel free to use results from previous problems on this round to prove a later problem 
(that is, you can use Problem 2 to prove Problem 3, but not vice versa). 
You do not need to have solved the earlier problem to cite its result.
You may also use results from the morning's Practice Power Round.

\section{Review}

\begin{definition}
An $n \times n$ \textit{magic square} is an $n \times n$ grid with integers in each cell such that
the sum of the integers in any row, the sum of the integers in any column, and the sum of the integers
in either of the two long diagonals are all equal. This sum is called the \textit{magic sum}.
\end{definition}

\begin{definition}
A \textit{standard} $n \times n$ magic square is one that uses the integers from $1$ through $n^2$, inclusive, each once.
\end{definition}

\begin{problem}[1 point]
Determine all $1 \times 1$ standard magic squares.
\end{problem}

\begin{problem}[5 points]
Determine the magic sum for an $n \times n$ magic square, where $n$ is any positive intger.
(Hint: You may find the formula $1 + 2 + 3 + \cdots + k = \frac{k(k+1)}{2}$ useful.)
What does your formula give for the magic sum for each of $n = 1, 2, 3, 4$?
Are these consistent with what you found this morning?
(If not, go back and check your work for your formula.)
Is your expression always guaranteed to be an integer? Why or why not?
\end{problem}


\section{$3 \times 3$ Magic Squares}

\begin{problem}[3 points]
Fill in the missing entries below to make a magic square, or prove that it is impossible to do so.
\begin{center}
\begin{tabular}{|c|c|c|}
\hline
20 & \phantom{00} & \phantom{00} \\ \hline
 & 17 & \\ \hline
 10 & 28 & \\ \hline
\end{tabular}
\end{center}
\end{problem}

\begin{problem}[15=1+3+2+3+3+3 points]
In this problem, you will construct a standard $3 \times 3$ magic square.
\begin{enumerate}[label=(\alph*)]
\item What is the magic sum?

\item List all the ways you can add three distinct numbers from the set $\{1, 2, 3, 4, 5, 6, 7, 8, 9\}$ to get the magic sum.
(Order doesn't matter here, so $4+5+6$ is the same as $5+6+4$.)

\item How many rows, columns, and diagonals are there? How many ways did you find in part (b)?
These two numbers should be the same. What does that mean?

\item Which number has to go in the center? Why?

\item Which numbers have to go in the corners? Why?

\item Draw a complete, standard $3 \times 3$ magic square.

\end{enumerate}
\end{problem}

\begin{problem}[12=4+4+4 points]
For each of the following, find a $3 \times 3$ magic square whose entries are the given numbers, or prove that it is impossible to do so.
\begin{enumerate}[label=(\alph*)]
\item 2, 3, 4, 5, 6, 7, 8, 9, 10

\item 2, 4, 6, 8, 10, 12, 14, 16, 18

\item 1, 3, 4, 5, 6, 7, 8, 9, 10
\end{enumerate}
\end{problem}

\section{Magic Rectangles}

\begin{definition}
An $m \times n$ \textit{magic rectangle} is a grid with $m$ rows and $n$ columns and intgers in each cell such that
the sum of the integers in each row is the same, and the sum of the integers in each column is the same,
but these two sums do not necessarily have to be the same.
The first is called the \textit{row sum} and the second is called the \textit{column sum}.
\end{definition}

\begin{definition}
A \textit{standard} $m \times n$ magic rectangle is one that uses the integers from $1$ through $m\times n$, inclusive, each once.
\end{definition}

\begin{problem}[8=2+2+4 points]
In this problem, you will construct a standard $4 \times 2$ magic rectangle.
\begin{enumerate}[label=(\alph*)]
\item Using the sum of the entries, determine the row sum and the column sum.

\item List all the ways you can add two of the possible numbers to obtain the row sum.

\item Draw a completed standard $4 \times 2$ magic rectangle.
\end{enumerate}
\end{problem}

\begin{problem}[8=4+4 points]
For each of the following, find a standard magic rectangle with the given dimensions or prove that it is impossible to do so.
\begin{enumerate}[label=(\alph*)]
\item $8 \times 2$

\item $8 \times 3$
\end{enumerate}
\end{problem}

\begin{problem}[6 points]
Show that it is impossible to construct a standard $m \times n$ rectangle if $m+n$ is odd.
\end{problem}

\end{document}
