\documentclass[11pt]{article}
\usepackage[paperwidth=8.5in, paperheight=11in]{geometry}

\usepackage{subfig}
\usepackage{../tjimo}

\newcommand{\sevenpoints}{Time limit: 45 minutes.}
\newcommand{\righthead}{\fdbox{Round}{Power Solutions}}

%\begin{comment}
\def\answer{\comment}
\def\solution{\comment}
\renewcommand{\righthead}{\fdbox{Round}{Power}}
%\end{comment}

\renewenvironment{problem}{\begin{problems}}{\end{problems}\vspace{5pt}}

\begin{document}

%\setlength{\belowcaptionskip}{-50pt}
\setlength{\parindent}{0pt}

Unlike the other rounds, just getting the answer right is not enough on the Power Round. 
Make sure you explain your answer and use words to describe how you arrived at your answer. 
In the words of middle school math teachers across the nation -- no work, no credit! \newline

Feel free to use results from previous problems on this round to prove a later problem 
(that is, you can use Problem 2 to prove Problem 3, but not vice versa). 
You do not need to have solved the earlier problem to cite its result.
You may also use results from the morning's Practice Power Round. \newline

This Power Round is divided into three sections. The first is a short review from the morning's 
Practice Power round, and is recommended to be completed first. 
The remaining two sections are on $3 \times 3$ magic squares and more on magic rectangles, 
and can be completed independently of each other.  They are equally weighted in terms points.
It might be wise to split up your team after finishing section 1, with one half working on section 2 and the other half working on section 3.

\section{Review}

\begin{definition}
An $n \times n$ \textit{magic square} is an $n \times n$ grid with positive integers in each cell such that
the sum of the integers in any row, the sum of the integers in any column, and the sum of the integers
in either of the two long diagonals are all equal. This sum is called the \textit{magic sum}.
\end{definition}

\begin{definition}
A \textit{standard} $n \times n$ magic square is one that uses the integers from $1$ through $n^2$, inclusive, each once.
\end{definition}

\begin{problem}[1 point]
Determine all standard $1 \times 1$ magic squares.
\end{problem}

\begin{solution}
By definition, a standard $1 \times 1$ magic square is a $1 \times 1$ grid that uses the number $1$.
Obviously, there is only one possible such magic square.
\begin{center}
\begin{tabular}{|c|}
\hline
1 \\ \hline
\end{tabular}
\end{center}
\end{solution}

\begin{problem}[5=3+1+1 points]
In this problem, you will consider the magic sum for an arbitrary standard magic square.
\begin{enumerate}[label=(\alph*)]
\item Determine the magic sum for an $n \times n$ magic square, where $n$ is any positive integer.
(Hint: You may find the formula $1 + 2 + 3 + \cdots + k = \frac{k(k+1)}{2}$ useful.)

\item What does your formula give for the magic sum for each of $n = 1, 2, 3, 4$?
Are these consistent with what you found this morning?
(If not, go back and check your work for your formula.)

\item Is your expression always guaranteed to be an integer? Why or why not? (Hint: Consider when $n$ is odd
or even.)
\end{enumerate}
\end{problem}

\begin{solution}
\begin{enumerate}[label=(\alph*)]
\item We use the same double counting approach from the Practice Power Round. The entries for a standard
$n \times n$ magic square are the integers from $1$ through $n^2$, inclusive, the sum of which is
$1 + 2 + \cdots + n^2 = \frac{n^2(n^2+1)}{2}$. (Here, we substituted $n^2$ for $k$ in the hint's formula.)
Suppose the magic sum is $S$. Then the sum of the entries is also the sum of the $n$ rows, which is $nS$.
Hence $nS = \frac{n^2(n^2+1)}{2}$, so $S = \boxed{\frac{n(n^2+1)}{2}}$.

\item This gives a magic sum of $\frac{1(2)}{2} = 1$ for $n = 1$, $\frac{2}{5}{2} = 5$ for $n = 2$,
$\frac{3}{10}{2} = 15$ for $n = 3$, and $\frac{4(17)}{2} = 34$ for $n = 4$, all of which are consistent
with those from the Practice Power Round.

\item $\boxed{\text{Yes}}$, $n$ and $n^2+1$ have different parity, so at least one of them will be divisible by $2$.
\end{enumerate}
\end{solution}


\section{$3 \times 3$ Magic Squares}

\begin{problem}[3 points]
Fill in the missing entries below to make a magic square, or prove that it is impossible to do so.
\begin{center}
\begin{tabular}{|c|c|c|}
\hline
20 & \phantom{00} & \phantom{00} \\ \hline
 & 17 & \\ \hline
 10 & 28 & \\ \hline
\end{tabular}
\end{center}
\end{problem}

\begin{solution}
Suppose the bottom-right entry (last row and last column) is $x$. Then the diagonal sum is $20+17+x = 37+x$,
but the sum of the entries in the last row is $10+28+x = 38+x$. The fact that the magic sum is the same
requires $37+x = 38+x$, or $0 = 1$, which is impossible.
\end{solution}


\begin{problem}[15=1+3+2+3+3+3 points]
In this problem, you will construct a standard $3 \times 3$ magic square.
\begin{enumerate}[label=(\alph*)]
\item What is the magic sum?

\item List all the ways you can add three distinct numbers from the set $\{1, 2, 3, 4, 5, 6, 7, 8, 9\}$ to get the magic sum.
(Order doesn't matter here, so $4+5+6$ is the same as $5+6+4$.)

\item How many rows, columns, and diagonals are there? How many ways did you find in part (b)?
These two numbers should be the same. What does that mean?

\item Which number has to go in the center? Why?

\item Which numbers have to go in the corners? Why?

\item Draw a complete, standard $3 \times 3$ magic square.
\end{enumerate}
\end{problem}

\begin{solution}
\begin{enumerate}[label=(\alph*)]
\item As computed above, the magic sum is $\boxed{15}$.

\item The possible ways to add three distinct numbers between $1$ and $9$, inclusive, to get $15$ are $1+5+9$, 
$1+6+8$, $2+4+9$, $2+5+8$, $2+6+7$, $3+4+8$, $3+5+7$, and $4+5+6$.

\item There are $3$ rows, $3$ columns, and $2$ diagonals. We found $\boxed{8}$ possible ways in part (b), which
equals $3+3+2$. This is the minimum number of ways necessary for there to possibly exist a magic square.
If a $3 \times 3$ standard magic square exists, it must use each of these ways exactly once.
Recall from the Practice Power Round that it was impossible to construct a standard $2 \times 2$ magic square
because there were only $2$ ways while a minimum of $6$ were needed.

\item The center entry (second row, second column) is included in one row, one column, and two diagonals.
The only entry that occurs in $4$ of the $8$ possible ways to sum to $15$ is $\boxed{5}$ (in $1+5+9$, $2+5+8$,
$3+5+7$, and $4+5+6$), so it must go in the middle.

\item Each corner number occurs in one row, one column, and one diagonal, so it needs to occur in three sums.
These numbers are $\boxed{2, 4, 6, \text{and } 8}$.

\item The remaining odd numbers $1, 3, 5$, and $7$ must go along the four remaining edge squares.
Using this, we can now construct a standard $3 \times 3$ magic square. One such example is shown below.
\begin{center}
\begin{tabular}{|c|c|c|}
\hline
8 & 1 & 6 \\ \hline
3 & 5 & 7 \\ \hline
4 & 9 & 2 \\ \hline
\end{tabular}
\end{center}
\end{enumerate}
\end{solution}


\begin{problem}[12=4+4+4 points]
For each of the following, find a $3 \times 3$ magic square whose entries are the given numbers, or prove that it is impossible to do so.
\begin{enumerate}[label=(\alph*)]
\item 2, 3, 4, 5, 6, 7, 8, 9, 10

\item 2, 4, 6, 8, 10, 12, 14, 16, 18

\item 1, 3, 4, 5, 6, 7, 8, 9, 10
\end{enumerate}
\end{problem}

\begin{solution}
\begin{enumerate}[label=(\alph*)]
\item The easiest way to do this is to take any standard $3 \times 3$ magic square and increase every entry by $1$.
Using the example above gives the result below.
\begin{center}
\begin{tabular}{|c|c|c|}
\hline
9 & 2 & 7 \\ \hline
4 & 6 & 8 \\ \hline
5 & 10 & 3 \\ \hline
\end{tabular}
\end{center}

\item Similar to part (a), in this case we multiply each element of the standard magic square by $2$.
\begin{center}
\begin{tabular}{|c|c|c|}
\hline
16 & 2 & 12 \\ \hline
6 & 10 & 14 \\ \hline
8 & 18 & 4 \\ \hline
\end{tabular}
\end{center}

\item We will prove that no such magic square exists. The sum of the given entries is $1+3+4+5+6+7+8+9+10=53$.
As we have seen many times, if the magic sum is $S$, then the sum of entries is $3S$. Then $3S = 53$, or 
$S = \frac{53}{3}$, which is not an integer. But all entries are integers, so this is impossible!
\end{enumerate}
\end{solution}


\section{Magic Rectangles}

\begin{definition}
An $m \times n$ \textit{magic rectangle} is a grid with $m$ rows and $n$ columns and positive integers in each cell such that
the sum of the integers in each row is the same and the sum of the integers in each column is the same,
but these two sums do not necessarily have to be the same.
The first is called the \textit{row sum} and the second is called the \textit{column sum}.
\end{definition}

\begin{definition}
A \textit{standard} $m \times n$ magic rectangle is one that uses the integers from $1$ through $m\times n$, inclusive, each once.
\end{definition}

\begin{problem}[3 points]
Fill in the missing entries below to make a standard magic rectangle, or prove that it is impossible to do so.
\begin{center}
\begin{tabular}{|c|c|c|c|c|}
\hline
6 & \phantom{7} & 8 & \phantom{9} & 10 \\ \hline
13 & \phantom{3} & 1 & 11 & \phantom{12} \\ \hline
\phantom{5} & 14 & 15 & 4 & \phantom{2} \\ \hline
\end{tabular}
\end{center}
\end{problem}

\begin{solution}
The third column is already completed, so the column sum is $8+1+15 = 24$. Then the missing entry in the
fourth column must be $24-11-4=9$ and the missing entry in the first column must be $24-6-13=5$.
At this point, we can either simply use logic with the remaining $4$ entries, which must be
$2$, $3$, $7$, and $12$, in some order, since this is a standard magic rectangle, or we can compute
the row sum via double counting, which turns out to be $40$. In either case, there is only one way to complete the magic rectangle.
\begin{center}
\begin{tabular}{|c|c|c|c|c|}
\hline
6 & 7 & 8 & 9 & 10 \\ \hline
13 & 3 & 1 & 11 & 12 \\ \hline
5 & 14 & 15 & 4 & 2 \\ \hline
\end{tabular}
\end{center}
\end{solution}


\begin{problem}[8=2+2+4 points]
In this problem, you will construct a standard $4 \times 2$ magic rectangle.
\begin{enumerate}[label=(\alph*)]
\item Using the sum of the entries, determine the row sum and the column sum.

\item List all the ways you can add two of the possible numbers to obtain the row sum.

\item Draw a complete, standard $4 \times 2$ magic rectangle.
\end{enumerate}
\end{problem}

\begin{solution}
\begin{enumerate}[label=(\alph*)]
\item The sum of the integers from $1$ through $8$ is $1+2+3+4+5+6+7+8=36$. There are $4$ rows, so the
row sum is $\frac{36}{4} = \boxed{9}$. There are $2$ columns, so the column sum is $\frac{36}{2} = \boxed{18}$.

\item There are four possible ways, up to order: $1+8$, $2+7$, $3+6$, and $4+5$.

\item Using the above, along with considerations for the column sum, we can construct a standard $4 \times 2$
magic rectangle. One such example is shown below.
\begin{center}
\begin{tabular}{|c|c|}
\hline
1 & 8 \\ \hline
7 & 2 \\ \hline
6 & 3 \\ \hline
4 & 5 \\ \hline
\end{tabular}
\end{center}
\end{enumerate}
\end{solution}


\begin{problem}[6=3+2+1 points]
In this problem, you will investigate the row and column sums for arbitrary magic rectangles.
\begin{enumerate}[label=(\alph*)]
\item Determine the row sum and the column sum for a standard $m \times n$ magic rectangle.
(Hint: You may find the formula given in the hint for problem 2 useful.)

\item What do your formulas give for a $4 \times 2$ magic rectangle? What about a $4 \times 3$ magic rectangle?
Are these correct?

\item Is your expression always guaranteed to be an integer? Why or why not? What are the implications of this?
\end{enumerate}
\end{problem}

\begin{solution}
\begin{enumerate}[label=(\alph*)]
\item Once again, we will use the method of double counting. A standard $m \times n$ magic rectangle
uses the integers from $1$ throguh $mn$ each once, so the sum of all the entries is 
$1 + 2 + 3 + \cdots + mn = \frac{mn(mn+1)}{2}$. There are $m$ rows, so the row sum is $\boxed{\frac{n(mn+1)}{2}}$.
There are $n$ columns, so the column sum is $\boxed{\frac{m(mn+1)}{2}}$.

\item For a $4 \times 2$ magic rectangle the row sum is $\frac{2(9)}{2} = \boxed{9}$ and the column sum is
$\frac{4(9)}{2} = \boxed{18}$. These are consistent with the previous problem.

For a $4 \times 3$ magic rectangle, the formula for the row sum gives $\frac{3(9)}{2} = 13.5$, which is not
an integer, which makes no sense since all the entries are integers. This means that no standard $4 \times 3$
magic rectangle exists.

\item $\boxed{\text{No}}$, the formulas do not always give an integer, as demonstrated by the standard
$4 \times 3$ magic rectangle in the previous part. This means that there are some dimensions for which
it is impossible to construct a standard magic rectangle.
\end{enumerate}
\end{solution}


\begin{problem}[8=4+4 points]
For each of the following, find a standard magic rectangle with the given dimensions or prove that it is impossible to do so.
\begin{enumerate}[label=(\alph*)]
\item $8 \times 2$

\item $8 \times 3$
\end{enumerate}
\end{problem}

\begin{solution}
\begin{enumerate}[label=(\alph*)]
\item We can construct this very similarly to how we found a standard $4 \times 2$ magic rectangle before.
One such example is shown below.
\begin{center}
\begin{tabular}{|c|c|}
\hline
1 & 16 \\ \hline
15 & 2 \\ \hline
3 & 14 \\ \hline
13 & 4 \\ \hline
12 & 5 \\ \hline
6 & 11 \\ \hline
10 & 7 \\ \hline
8 & 9 \\ \hline
\end{tabular}
\end{center}

\item The sum of the integers from $1$ through $8 \cdot 3 = 24$ is $\frac{24(25)}{2} = 150$. There are $8$
rows, so the row sum would have to be $\frac{150}{8} = 18.75$, which is not an integer. But all the entries
are integers, so this is impossible, meaning that it is impossible to construct a standard $8 \times 3$ magic
rectangle.
\end{enumerate}
\end{solution}


\begin{problem}[5 points]
Show that it is impossible to construct a standard $m \times n$ rectangle if $m+n$ is odd.
\end{problem}

\begin{solution}
If $m+n$ is odd, then one of $m, n$ is odd and the other is even. Without loss of generality, suppose
$m$ is odd and $n$ is even. The sum of all the entries is $1 + 2 + 3 + \cdots + mn = \frac{mn(mn+1)}{2}$.
There are $n$ columns, meaning that the column sum would be $\frac{m(mn+1)}{2}$. However, since $m$ is odd
and $n$ is even, $mn + 1$ is also odd, so the numerator $m(mn+1)$ is the product of two odd numbers, which
cannot be divisible by $2$. Hence this is not an integer, so it is impossible to construct a standard
$m \times n$ rectangle, as desired.
\end{solution}

\end{document}
