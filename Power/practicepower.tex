\documentclass[11pt]{article}
\usepackage[paperwidth=8.5in, paperheight=11in]{geometry}

\usepackage{subfig}
\usepackage{../tjimo}

\newcommand{\sevenpoints}{Time limit: 30 minutes.}
\newcommand{\righthead}{\fdbox{Round}{Practice Power}}

\begin{comment}
\def\answer{\comment}
\def\solution{\comment}
\end{comment}

\renewenvironment{problem}{\begin{problems}}{\end{problems}\vspace{5pt}}

\begin{document}

%\setlength{\belowcaptionskip}{-50pt}
\setlength{\parindent}{0pt}

\section{Introduction}

Unlike the other rounds, just getting the answer right is not enough on the Power Round. 
Make sure you explain your answer and use words to describe how you arrived at your answer. 
In the words of middle school math teachers across the nation -- no work, no credit! \newline

Feel free to use results from previous problems on this round to prove a later problem 
(that is, you can use Problem 2 to prove Problem 3, but not vice versa). 
You do not need to have solved the earlier problem to cite its result.

\section{Basics}

\begin{definition}
An $n \times n$ \textit{magic square} is an $n \times n$ grid with integers in each cell such that
the sum of the integers in any row, the sum of the integers in any column, and the sum of the integers
in either of the two long diagonals are all equal. This sum is called the \textit{magic sum}.
\end{definition}

\begin{problem}[3=1+1+1 points]
Determine whether each of the following is a magic square. If it is, determine the magic sum. If not, find two sums that are different.

\begin{figure}[H]
\centering
\subfloat[]{
\begin{tabular}{|c|c|}
\hline
1 & 4 \\ \hline
2 & 3 \\ \hline
\end{tabular}
}
\qquad
\subfloat[]{
\begin{tabular}{|c|c|}
\hline
17 & 17 \\ \hline
17 & 17 \\ \hline
\end{tabular}
}
\qquad
\subfloat[]{
\begin{tabular}{|c|c|c|}
\hline
4 & 2 & 9 \\ \hline
3 & 7 & 5 \\ \hline
8 & 6 & 1 \\ \hline
\end{tabular} 
}
\end{figure}
\end{problem}

\begin{definition}
A \textit{standard} $n \times n$ magic square is one that uses the integers from $1$ through $n^2$, inclusive, each once.
\end{definition}

\begin{problem}[4 points]
Fill in the missing entries below (replicated on the answer sheet) so that the result is a standard $4 \times 4$ magic square.

\begin{center}
\begin{tabular}{|c|c|c|c|}
\hline
16 & 2 & 3 & 13 \\ \hline
\phantom{5} & \phantom{11} & \phantom{10} & \phantom{8} \\ \hline
\phantom{9} & 7 & 6 & \phantom{12} \\ \hline
\phantom{4} & \phantom{14} & 15 & 1 \\ \hline
\end{tabular}
\end{center}
\end{problem}

\begin{problem}[5=1+1+1+2 points]
This problem will guide you through computing the magic sum for a standard $3 \times 3$ magic square.

\begin{enumerate}[label=(\alph*)]
\item What are the nine numbers used for the entries of this standard $3 \times 3$ magic square?

\item What is the sum of these $9$ numbers?

\item Let the magic sum be $S$. Recall that this equals the sum of the entries in \textit{any} row. How many rows are there? 
In terms of $S$, what is the sum of all these rows?

\item Determine the value of $S$.
\end{enumerate}
\end{problem}

This method is called \textit{double counting}, where you count the same thing in two different ways and equate them.
In this case, you calculated the sum of all the entries in two different ways.

\begin{problem}[3 points]
Determine the magic sum for a standard $4 \times 4$ magic square.
\end{problem}


\section{$2 \times 2$ Magic Squares}

\begin{problem}[6=2+2+2 points]
In this problem, you will try to construct a standard $2 \times 2$ magic square.
\begin{enumerate}[label=(\alph*)]
\item What would be the magic sum of a standard $2 \times 2$ magic square?

\item List all the ways you can add two distinct numbers from the set $\{1, 2, 3, 4\}$ to get the magic sum.

\item Fill in the entries for a standard $2 \times 2$ magic square, or explain why this is impossible.
\end{enumerate}
\end{problem}

\begin{problem}[4 points]
Consider the partially-completed $2 \times 2$ \textit{non-standard} magic square.
\begin{center}
\begin{tabular}{|c|c|}
\hline
17 & \phantom{17} \\ \hline
 & \\ \hline
\end{tabular}
\end{center}
Fill in the missing entries, justifying each step and explaining why your solution is the only possible one.
\end{problem}

\begin{problem}[4 points]
Find all possible $2 \times 2$ magic squares with integer entires. The one above should be included.
\end{problem}

\section{Magic Rectangles}

\begin{definition}
An $m \times n$ \textit{magic rectangle} is a grid with $m$ rows and $n$ columns and intgers in each cell such that
the sum of the integers in each row is the same, and the sum of the integers in each column is the same,
but these two sums do not necessarily have to be the same.
The first is called the \textit{row sum} and the second is called the \textit{column sum}.
\end{definition}

\begin{problem}[3 points]
Show that any magic rectangle whose row sum equals the column sum must, in fact, be a magic square.
\end{problem}

\end{document}
